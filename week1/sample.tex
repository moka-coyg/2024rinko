\documentclass{jarticle}
% 図をFig. に変えておく
\renewcommand{\figurename}{Fig.}

% 画像
\usepackage[dvipdfmx]{graphicx}
% リンク
\usepackage[dvipdfmx]{hyperref}
% URL
\usepackage{url}
% 目次
\usepackage{pxjahyper}
% フォント関連
\usepackage{amsfonts}

\usepackage{amsmath, amssymb}
\usepackage{mathrsfs}	% for \mathscr{}
\usepackage{bm}
% ソースコード表示のための設定
\usepackage{listings,jlisting}
\lstset{
  basicstyle={\ttfamily},
  identifierstyle={\small},
  commentstyle={\smallitshape},
  keywordstyle={\small\bfseries},
  ndkeywordstyle={\small},
  stringstyle={\small\ttfamily},
  frame={tb},
  breaklines=true,
  columns=[l]{fullflexible},
  numbers=left,
  xrightmargin=0zw,
  xleftmargin=3zw,
  numberstyle={\scriptsize},
  stepnumber=1,
  numbersep=1zw,
  lineskip=-0.5ex
}

% 余白の設定
\usepackage[margin=20truemm]{geometry}

\begin{document}
\title{1年生実習 第1週}
\author{B5研究室}
\date{2024年6月19日}
\maketitle

\section{実習概要}

これから5週間にわたり,実習を行います.各週の実習内容は以下の通りです.

\begin{description}
  \item[第1週-第3週] Python入門
  \item[第4週] Leap Motionを用いて手書き文字を取得
  \item[第5週] 機械学習による手書き文字の分類
\end{description}

実行環境はGoogle Colaboratoryを使用することを想定しています.他の環境でももちろん大丈夫です.

\section{Python入門}
\subsection{入出力}

Pythonで入出力を行うには,\texttt{input()}関数と\texttt{print()}関数を用います.
プログラムの例をソースコード\ref{std_in_out}に示します.
\begin{lstlisting}[caption=入出力,label=std_in_out]
# 標準入力
name = input("名前を入力してください: ")

# 標準出力
print("こんにちは," + name + "さん")

# ほかの書き方
print("こんにちは,{}さん".format(name))
print(f"こんにちは,{name}さん")
\end{lstlisting}

input文で文字列を入力した場合,その文字列は文字列型として扱われます.数字を入力したい場合は,\texttt{int()}関数や\texttt{float()}関数を用いて数値型に変換する必要があります.(ソースコード\ref{input_number})
\begin{lstlisting}[caption=数値の入力,label=input_number]
  # 数値の入力
  price = input("価格を入力してください: ")
  print(type(price))  # <class 'str'> <- これは文字列型

  price_include_tax = price * 1.1 # エラー

  # 数値型に変換
  price = int(price)
  print(type(price))  # <class 'int'> <- これは数値型

  price_include_tax = price * 1.1 # 正しい結果が得られる
  print("税込み価格は" + int(price_include_tax) + "円です")
  \end{lstlisting}

\subsection{四則演算}
Pythonで四則演算を行うには,ソースコード\ref{calculation}のように記述します.
\begin{lstlisting}[caption=四則演算,label=calculation]
  a = 10
  b = 3
  print(a + b)  # 足し算
  print(a - b)  # 引き算
  print(a * b)  # 掛け算
  print(a / b)  # 割り算
  print(a // b) # 割り算(整数部のみ)
  print(a % b)  # 割り算(余り)
  print(a ** b) # べき乗
  print(abs(a)) # 絶対値
\end{lstlisting}

\subsection{分岐処理}
変数の値によって処理を分岐させる場合は,\texttt{if}文を用います.ソースコード\ref{if}に例を示します.
\begin{lstlisting}[caption=if文,label=if]
  age = int(input("数字を入力してください: "))
  if a > 0:
    print("aは正の数です")
  elif a == 0:
    print("aは0です")
  else:
    print("aは負の数です")
\end{lstlisting}

\texttt{if}文は,条件分岐を行うための文です.\texttt{if}文の条件が真の場合,その処理を実行します.\texttt{else}文は,\texttt{if}文の条件が偽の場合に実行される処理を記述します.\texttt{else}文は,\texttt{if}文の後に記述します.
\texttt{elif}文は,\texttt{if}文と\texttt{else}文の間に挟まれる条件分岐文です.\texttt{elif}文は,複数の条件を順番に評価し,最初に真となった条件の処理を実行します.\texttt{elif}文は,\texttt{if}文の条件が偽で,かつ自身の条件が真の場合に処理を実行します.

\texttt{if}文の条件式には,比較演算子や論理演算子を用いることができます.
比較演算子とは,2つの値を比較するための演算子です.例えば,\texttt{a > b}は,\texttt{a}が\texttt{b}より大きい場合に\texttt{True}を返します.論理演算子とは,複数の条件を組み合わせるための演算子です.例えば,\texttt{a > 0 and a < 10}は,\texttt{a}が0より大きくかつ10より小さい場合に\texttt{True}を返します.以下に比較演算子と論理演算子の一覧を示します.

\begin{itemize}
  \item 比較演算子:
  \begin{itemize}
    \item \texttt{==} : 等しい
    \item \texttt{!=} : 等しくない
    \item \texttt{>} : より大きい
    \item \texttt{<} : より小さい
    \item \texttt{>=} : 以上
    \item \texttt{<=} : 以下
  \end{itemize}
  \item 論理演算子:
  \begin{itemize}
    \item \texttt{and} : 論理積 (両方の条件が真の場合に真)
    \item \texttt{or} : 論理和 (どちらかの条件が真の場合に真)
    \item \texttt{not} : 否定 (条件の真偽を反転)
  \end{itemize}
\end{itemize}

\subsection{ループ処理}
繰り返して処理を行う場合は,\texttt{for}文や\texttt{while}文を用います.
\texttt{for}文は,リストやタプルなどの要素を1つずつ取り出して処理を行う場合に使用します.
\texttt{while}文は,条件が真の間,処理を繰り返す場合に使用します.ソースコード\ref{loop}に例を示します.
\begin{lstlisting}[caption=ループ処理,label=loop]
  # for文
  for i in range(5):
    print(i)

  # while文
  i = 0
  while i < 5:
    print(i)
    i += 1 # i = i + 1と同じ
\end{lstlisting}

range関数は,指定した範囲の整数を生成する関数です.range関数は,for文において,繰り返し回数を指定する際によく使用されます.range関数の使い方を以下に示します.
\begin{itemize}
  \item \texttt{range(n)} : 0からn-1までの整数を生成
  \item \texttt{range(a, b)} : aからb-1までの整数を生成
  \item \texttt{range(a, b, c)} : aからb-1までの整数をc刻みで生成
\end{itemize}

また,\texttt{break}文や\texttt{continue}文を用いることで,ループ処理を制御することができます.
\texttt{break}文は,ループ処理を中断し,ループから抜け出す場合に使用します.
\texttt{continue}文は,ループ処理を中断し,次の繰り返しに移る場合に使用します.
ソースコード\ref{break_continue}に例を示します.
\begin{lstlisting}[caption=break文とcontinue文,label=break_continue]
  while(True): # 無限ループ
    number = int(input("数字を入力してください: "))
    if number == 0:
      continue # 0のときは処理をスキップ
    if number < 0:
      break # 負の数のときはループを抜ける
    print(number)
\end{lstlisting}

無限ループは,ソースコード\ref{break_continue}のように,\texttt{while(True):}で表現することができます.

\subsection{リストとタプル}
リストとタプルとは,複数の要素をまとめて扱うためのデータ構造です.
リストは,\texttt{[]}で囲まれた要素の集まりであり,要素の追加や削除が可能です.
タプルは,\texttt{()}で囲まれた要素の集まりであり,要素の追加や削除ができません.
同様の考え方として,C言語における配列があります.c言語における配列は,要素数が固定されており,要素の追加や削除ができません.一方,リストとタプルは要素数が可変であるという特徴があります.
ソースコード\ref{list_tuple}に例を示します.
\begin{lstlisting}[caption=リストとタプル,label=list_tuple]
  # リスト
  fruits = ["apple", "banana", "orange"]
  print(fruits[0]) # apple
  fruits.append("grape") # 要素の追加
  print(fruits) # ["apple", "banana", "orange", "grape"]

  # タプル
  fruits = ("apple", "banana", "orange")
  print(fruits[0]) # apple
  fruits.append("grape") # エラー
\end{lstlisting}

リストに要素を追加するには,\texttt{append()}メソッドを用います.同様に,削除をするには,\texttt{remove()}メソッドを用います.リストの操作として重要なものにスライスがあります.スライスとは,リストやタプルから一部の要素を取り出す操作です.他にも,挿入など,リストに対して様々な操作が可能です.ソースコード\ref{list_example}に例を示します.
\begin{lstlisting}[caption=リストの操作,label=list_example]
  fruits = ["apple", "banana", "orange"]
  fruits.insert(1, "grape") # 1番目に挿入
  print(fruits) # ["apple", "grape", "banana", "orange"]

  fruits.remove("banana") # bananaを削除
  print(fruits) # ["apple", "grape", "orange"]

  print(fruits[1:]) # ["grape", "orange"] # 1番目以降の要素を取得(スライス)
\end{lstlisting}

リストとfor文を組み合わせることで,リストの要素を1つずつ取り出して処理を行うことができます.ソースコード\ref{list_for}に例を示します.
\begin{lstlisting}[caption=リストとfor文,label=list_for]
  fruits = ["apple", "banana", "orange"]
  for fruit in fruits:
    print(fruit) # apple, banana, orangeが順に表示される
\end{lstlisting}

\section{演習}
\begin{enumerate}
  \item 実行すると,Hello, World!と表示するプログラムを作成してください.
  \item 12345+23456を計算して結果を表示するプログラムを作成してください.
  \item 12345を7で割った余りを表示するプログラムを作成してください.
  \item 整数値を入力し,その入力値を表示するプログラムを作成してください.
  \item 整数値を入力し,その入力値を3倍した計算結果を表示するプログラムを作成してください.
  \item 整数値を2つ入力し,それらの値の和,差,積,商と余りを求めるプログラムを作成してください.
  \item 整数値を入力し,値が0ならzeroと表示するプログラムを作成してください.
  \item 整数値を入力し,値が正ならpositive,負ならnegative,0ならzeroと表示するプログラムを作成してください.
  \item 整数値を入力し,その値を絶対値にして表示するプログラムを作成してください.ただし,abs()関数を使用しないでください.
  \item Hello World!を10回繰り返して表示するプログラムを作成してください.
  \item 整数値を入力し,その値の回数だけHello World!を繰り返して表示するプログラムを作成してください.
  \item 整数値を入力し,0から入力値まで数を1ずつ増やして表示するプログラムを作成してください.
  \item 整数値を入力し,入力値から0まで数を1ずつ減らして表示するプログラムを作成してください.
  \item 整数値を入力し,0から入力値を超えない値まで2ずつ増やして表示するプログラムを作成してください.
  \item \texttt{teachers = ["Fukumi", "S\_Ito", "M\_Ito"]}というリストを作成し,リストの要素を1つずつ表示するプログラムを作成してください.
  \item teachersの要素を1つずつ表示するプログラムを作成してください.ただし,要素の表示順を逆にしてください.(ヒント:\texttt{reversed()}関数を使用する)
  \item teachersに \texttt{"Tokushima"}を追加してください.
  \item teachersから \texttt{"Tokushima"}を削除してください.
\end{enumerate}

\section{課題}
\subsection{FizzBuzz問題}
プログラミングの入門問題として,FizzBuzz問題がよく用いられます.
FizzBuzz問題とは,1から100までの数を順に表示するプログラムを作成する問題です.ただし,3の倍数の場合はFizz,5の倍数の場合はBuzz,3の倍数かつ5の倍数の場合はFizzBuzzと表示するプログラムを作成してください.1から30までの出力例を以下に示します.
\begin{verbatim}
  1
  2
  Fizz
  4
  Buzz
  Fizz
  7
  8
  Fizz
  Buzz
  11
  Fizz
  13
  14
  FizzBuzz
  16
  17
  Fizz
  19
  Buzz
  Fizz
  22
  23
  Fizz
  Buzz
  26
  Fizz
  28
  29
  FizzBuzz
\end{verbatim}

\subsection{調査課題}
以下に示すPythonの文や操作の内容について調べてください.調べた内容と,実行した例を報告してください.なお,int型,float型,str型,bool型については,実行例は不要です.
\begin{itemize}
  \item ディクショナリ
  \item pass文
  \item リスト内包表記
  \item int型
  \item float型
  \item str型
  \item bool型
\end{itemize}
\end{document}
